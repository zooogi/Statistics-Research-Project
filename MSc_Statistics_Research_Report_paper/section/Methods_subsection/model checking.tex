\subsubsection{Why Are Two Layers of Model Checking Necessary}
\label{subsec:wo Layers of Model Checking}
In the previous section, we derived the exponential survival model under a Gamma prior, demonstrated its closed-form posterior, and confirmed the agreement between analytical and MCMC-based computation. While this validates parameter estimation under the model’s assumptions, it does not guarantee that the model itself captures the key structural features of the observed data~\cite{62bfc978-09b1-3997-9776-380d0b45e9c2, gelman1995bayesian}.

In practice, the actual data-generating process may involve additional unobserved mechanisms~\cite{kalbfleisch2002statistical}. For example, there may be a fixed observation window applied to all individuals, shaping the censoring pattern in ways the current model does not explicitly represent. Ignoring such mechanisms can lead to systematic bias in parameter estimates and distort model predictions~\cite{stats5010006}.

Before drawing inferences or making interpretations, it is therefore necessary to examine both structural validity and adequacy of fit~\cite{62bfc978-09b1-3997-9776-380d0b45e9c2}. In Bayesian modelling, the primary purpose of model checking is to assess whether the proposed model is grounded in solid assumptions, both in terms of its structure and its fitting ability~\cite{https://doi.org/10.1002/ecm.1314}. Unlike traditional approaches that rely solely on goodness-of-fit metrics (such as the likelihood value or information criteria like WAIC or LOO)~\cite{cho2025nonlinear, https://doi.org/10.1002/ecm.1314}, model checking shifts the focus away from mere “score performance” on observed data and toward verifying whether the model captures key structural features from a generative perspective. 

As shown in Figure~\ref{flowchart}, this process reflects two fundamental validation tasks in Bayesian modelling.
\begin{enumerate}
    \item Is the model structure adequate to explain the observed data?
    \item To what extent can the model reproduce reality after inference?
\end{enumerate}
%%%%%%%%%%%%%%%%%%%
%%%%%%%%%%%%%%%流程图workflow%%%%%%%%%%
\begin{figure}[H]
\centering
\resizebox{0.52\linewidth}{!}{
\begin{tikzpicture}[
 scale=0.50,
  every node/.style={transform shape}, 
  node distance=11mm,
  box/.style      ={rectangle, draw, rounded corners, align=center,
                    minimum width=44mm, minimum height=9mm},
  decision/.style ={diamond, aspect=2.2, draw, align=center, inner sep=1.4pt},
  ->, >=Latex
]

%--- Main process node ------------------------------------------------------
\node[box]      (model)   {\textbf{Define/Revise a model}};

\node[box]      (checking1)  [below=of model] {\textbf{Simulation-based Model Checking} \\
(Simulate with $\lambda_0$; fit model; recover $\lambda_0$?)};

\node[decision] (pass1)   [below=10mm of checking1] {\textbf{Plausible?}};

\node[box]      (fit)     [below=of pass1]  {\textbf{Fit to real data}\\
(Compute posterior)};

\node[box]      (gen)     [below=of fit]    {\textbf{Posterior Predictive Model Checking}\\
(Generate fake data from Posterior)\\
(Compare fake vs real)};

\node[decision] (pass2)   [below=of gen] {\textbf{Adequate?}};

\node[box]     (report)  [below=of pass2] {\textbf{Report/Interpret}};

%--- connection ------------------------------------------------------------
\draw (model)   -- (checking1)
      (checking1)  -- (pass1)
      (pass1)   -- node[right]{Yes}(fit)
      (fit)     -- (gen)
      (gen) -- (pass2)
      (pass2) -- node[right]{Yes}(report);

%--- Loopback ------------------------------------
\draw[->] (pass1.west) -- ++(-30mm,0) |- (model.west) node[pos=0.23, left]{No};
%\draw[->] (pass1.west) to[out=180,in=180,looseness=1.3] node[left]{No} (model.west);
\draw[->] (pass2.east) -- ++ (30mm,0 )|- (model.east) node[pos=0.23,right]{No};
\end{tikzpicture}}
\caption{General Bayesian workflow}
\label{flowchart}
\end{figure}

This workflow clearly distinguishes two levels of model checking.

First is the \textbf{Simulation-based Model Checking}, which evaluates structural identifiability based on known parameters~\cite{10.1093/bioinformatics/btp358}. Before fitting any real data, we can simulate pseudo-datasets using a fixed value $\lambda_0$, and then re-fit the model using the same procedure. If we successfully “recover” $\lambda_0$, this suggests that the model structure is sound; failure to do so implies structural flaws in the model, rendering downstream inferences on real data invalid~\cite{10.1093/bioinformatics/btp358, pub.1044073403}.

Second is the \textbf{Posterior Predictive Checking}, which assesses posterior adequacy after fitting the model to real data~\cite{https://doi.org/10.1002/ecm.1314}. We draw samples from the posterior distribution and generate “fake data” to check whether the model can reproduce the statistical features of the observed data. The key idea is that if data generated under the posterior distribution differ significantly from real data in terms of censoring structure and event-time distribution, this suggests that the model fails to capture the true underlying process~\cite{62bfc978-09b1-3997-9776-380d0b45e9c2}. Bayesian methods naturally account for this uncertainty by incorporating the full posterior distribution.

To implement this, we design a simulation procedure to generate synthetic survival datasets that preserve the censoring mechanism of the real data (see Algorithm 1)~\cite{ashhad2025generatingaccuratesyntheticsurvival}. The algorithm proceeds by drawing a posterior sample $\lambda^*$ and defining a suitable time window $[a, 0]$ to simulate individuals' entry times and latent event durations. This allows us to generate a “virtual dataset” that matches the sample size and censoring structure of the original data, thereby enabling a rigorous comparison for posterior predictive model checking.
%%%%%%算法--------------
\begin{tcolorbox}[
  title  = Algorithm 1: Simulating a Fake Survival Dataset (e.g.Posterior predictive model checking),label={fake data},
   fonttitle  = \bfseries\footnotesize,
   fontupper = \footnotesize,
    %width = 0.9\textwidth,   
  box align = center, 
  colback = white,
  colframe=black,
   boxsep  = 3pt,
 left=4pt,right=4pt,top=5pt,bottom=4pt]
\textbf{Input}
\begin{itemize}[itemsep=1pt,parsep=0pt,topsep=2pt]
\item posterior samples $\{\lambda^{(s)}\}$ \hfill (obtained by fitting real data)
\item sample size $n$ \hfill (same size as the real data set)
\item start‑time range $[a,0]$ with $a<0$
\end{itemize}

\textbf{Algorithm}
\begin{enumerate}[itemsep=2pt,parsep=0pt,topsep=2pt]
  \item Choose a single $\lambda^\ast$ from posterior samples \hfill (e.g.\ posterior mean)
  \item For $i = 1,\dots,n$
          %\begin{minipage}[t]{\linewidth}
          \begin{enumerate}
            \item Draw latent duration: $y_i \sim \operatorname{Exp}(\lambda^\ast)$
            \item Draw start time: $T_i \sim \operatorname{Uniform}(a,0)$
            \item Compute leaving time: $t_i = T_i + y_i$
            \item Observed pair $(\text{time}_i,\text{event}_i)$ \[\begin{aligned}
\text{if } \quad t_i &< 0: &&&\text{(event occurred before now)}\\
  &&\quad  \text{event}_i=1, &&\text{(uncensored)}\\ 
  &&\quad \text{time}_i=y_i
   &\quad \\
&\text{else}: &&&\text{(event in the future)}\\
   &&\quad \text{event}_i=0, &&\text{(right censored)}\\ 
   &&\quad \text{time}_i=-T_i &&\text{(time already spent)}
   &\quad 
\end{aligned}\]
          \end{enumerate}
  \item Combine $(\text{time}_i,\text{event}_i)$ into a fake data set of size $n$.
\end{enumerate}
\end{tcolorbox}





\subsubsection{Model Checking via ECDF under Independent Censoring}
In model checking, to compare the overall distributional shapes of the real data and the simulated data, we use the empirical cumulative distribution function (ECDF) rather than histograms. Unlike histograms, ECDFs do not depend on subjective choices of bin widths and break points, thereby avoiding visual biases~\cite{berg2008data}. Moreover, an ECDF is a monotone right-continuous step function defined on $[0,\infty)$, which stably displays differences between samples over the entire time axis~\cite{arnold2011nonparametric, berg2008data}. Importantly, ECDFs can be plugged directly into distance statistics such as the Kolmogorov–Smirnov or Cramér–von Mises metrics~\cite{arnold2011nonparametric}, facilitating quantitative assessment of model fit.


%%%%%%%%%今天到这了,下面还没有添加文献!!!然后后面对应的需要修改的内容还没改。明天要改完他了!!!然后周六要跑那个高等线图!!!然后先写那个章节!!!周日要写intro!!!加标题
%%%或者周六先写个intro然后加上跑出来那个高等线图!!!
%%%周二才写那个似然函数的A那个可识别性。
%%%周二再继续改吧。然后我想在把结果写完之前发给他看看,然后让他给点建议。然后剩下最后一周要写完结果和conclusion了!!!

In survival data, the observed duration is determined jointly by the latent event time $T\ge 0$ and the censoring time $C\ge 0$. The observed quantity is
$$
Y=\min(T,C),\qquad \delta=\mathbf 1\{T\le C\}.
$$
That is, we observe the event time only when it is uncensored; otherwise, we observe the censoring time. Consequently, we split the data into two subsamples, an “event subsample’’ with $\delta=1$ and a “censored subsample’’ with $\delta=0$, and compute ECDFs for each subsample separately.

Under independent (non-informative) censoring, i.e., $T\perp C$, the two subsamples may be viewed as arising from two conditional distributions~\cite{fleming2013counting}
\begin{itemize}
    \item for the event subsample ($\delta=1$), the observed values $Y=T$ follow the conditional distribution $T\mid(T\le C)$;
    \item for the censored subsample ($\delta=0$), the observed values $Y=C$ follow the conditional distribution $C\mid(C<T)$.
\end{itemize}
Let the event-subsample size be $n_1=\sum_i \delta_i$ and the censored-subsample size be $n_0=\sum_i (1-\delta_i)$. The corresponding empirical distribution functions are~\cite{fleming2013counting, arnold2011nonparametric}
\begin{equation}
    \widehat H_{\text{event}}(t)
=\frac{1}{n_1}\sum_{i:\,\delta_i=1}\mathbf 1\{Y_i\le t\},\qquad
\widehat H_{\text{cens}}(t)
=\frac{1}{n_0}\sum_{i:\,\delta_i=0}\mathbf 1\{Y_i\le t\},\qquad t\ge 0.
\end{equation}
By the Glivenko–Cantelli theorem~\cite{tucker1959generalization}, as sample sizes grow, these ECDFs converge uniformly to their respective target distribution functions
\begin{align}
\widehat H_{\text{event}}(t) &\xrightarrow{\text{uniformly in } t} H_{\text{event}}(t) := \Pr(T \le t \mid T \le C) \\
\widehat H_{\text{cens}}(t) &\xrightarrow{\text{uniformly in } t} H_{\text{cens}}(t) := \Pr(C \le t \mid C < T)
\end{align}
In summary, by splitting the data into event and censored subsamples and plotting their empirical distribution functions $\widehat H_{\text{event}}$ and $\widehat H_{\text{cens}}$, we can evaluate model fit within a unified framework along both dimensions.

%\subsection{Posterior predictive CDF}
%\input{MSc_Statistics_Research_Report_paper/section/Methods_subsection/posterior predictive cdf }




