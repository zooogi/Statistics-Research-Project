After completing the Bayesian estimation for parameter $A$, a natural question arises: 
Does the extended model remain theoretically identifiable and consistent with the original exponential survival model? 
To address this, we first establish the identifiability of $(\lambda, A)$, and then examine how the joint posterior structure and marginalization of $A$ reflect consistency with the baseline model. 

\subsubsection{Identifiability}
Once the new parameter $A$ is introduced, the first question is whether the model remains identifiable. 
\begin{definition}
If the observed random variables are $(Y, \delta)$ whose distribution depends on parameters $(\lambda, A)$, denoted by
\begin{equation}
    \mathcal{L}(Y,\delta \mid \lambda,A),
\end{equation}
then the parameter pair $(\lambda, A)$ is identifiable if and only if
\begin{equation}
    \mathcal{L}(Y,\delta \mid \lambda_1,A_1) = \mathcal{L}(Y,\delta \mid \lambda_2,A_2)
\quad \Rightarrow \quad (\lambda_1,A_1) = (\lambda_2,A_2).
\end{equation}
In other words, no two distinct parameter pairs can induce the same distribution of the observations.
\end{definition}
\begin{proof}
    \begin{enumerate}
        \item \textbf{Identifiability of $A$}. The most direct approach is to examine the support of the observed data $Y$. For censored observations ($\delta=0$), the density is
         \begin{equation}
             f(Y=y,\delta=0 \mid \lambda,A) = \frac{1}{A} e^{-\lambda y}, \quad 0 \le y \le A.
         \end{equation}
         Its support is $[0, A]$, which is determined entirely by $A$ and independent of $\lambda$. Since the support is an intrinsic property of the distribution, two parameter pairs that generate the same distribution must share the same support, hence $A_1=A_2$. Thus, $A$ is identified from the support.
         \item \textbf{Identifiability of $\lambda$.}
         Once $A$ is fixed, all distributions share the same support $[0, A]$, so the support no longer distinguishes parameters. We then exploit the “shape’’ of the distribution via the ratio of event to censoring densities,
         \begin{equation}
             R(y) = \frac{f(Y=y,\delta=1 \mid \lambda,A)}{f(Y=y,\delta=0 \mid \lambda,A)} = \lambda (A-y), \quad 0<y<A.
         \end{equation}
         This ratio is linear in $y$, with slope $-\lambda$ and intercept $\lambda A$. Since $A$ has already been identified from the support, the slope uniquely determines $\lambda$.
    \end{enumerate}
    If two parameter pairs $(\lambda_1,A_1)$ and $(\lambda_2,A_2)$ induce the same observed distribution, then necessarily $A_1=A_2$, and hence $\lambda_1=\lambda_2$. Therefore, the parameter pair $(\lambda, A)$ is structurally identifiable in this model.
\end{proof}

\subsubsection{Grid-based Contour Analysis of the Joint Posterior}

To further examine how identifiability appears in finite-sample settings, we characterize the shape of the joint posterior distribution and visualize its structure numerically.

As shown in Equation~\eqref{A_post} of Section~\ref{A_bayes}, we adopt the same prior for $\lambda$ as in Section~\ref{指数模型贝叶斯推断过程} (the exponential model case), to ensure comparability and to facilitate the later marginalization check in Section~\ref{边际化章节}. For $A$, we use a uniform prior $\mathrm{Unif}(y_{\max}, y_{\max}+500)$, which is relatively broad but still consistent with the constraints discussed in Section~\ref{Impact of A}: the survey duration cannot extend indefinitely (e.g., 1000 months).

Under this setting, the log-posterior is
\begin{align}
\log p(\lambda, A \mid \mathcal D)
&= \text{const}
   + \Big(\sum_{i=1}^n \delta_i\Big)\log \lambda
   - \lambda \sum_{i=1}^n y_i \nonumber\\[6pt]
&\quad + \sum_{i:\,\delta_i=1}\log(A-y_i)
   - n \log A
   + \log \pi_\lambda(\lambda) \nonumber \\[6pt]
&\quad + \log \pi_A(A) + \log \mathbf 1\{A \ge \max_i y_i\}.
\end{align}
Since the model involves only two parameters $(\lambda, A)$, we approximate the posterior distribution by a regular two-dimensional grid. Specifically, we construct a grid of candidate points and evaluate the log-posterior $\log p(\lambda_i, A_j \mid \mathcal D)$ at each location. Exponentiating and normalizing then yields a discrete approximation of the posterior distribution
\begin{equation}
    w_{ij} \;\propto\; \exp\!\Bigl(\log p(\lambda_i, A_j \mid \mathcal D)\Bigr), \qquad \tilde w_{ij} \;=\; \frac{w_{ij}}{\sum_{i,j} w_{ij}} ,
\end{equation}
where $w_{ij}$ denotes the unnormalized posterior weight at $(\lambda_i,A_j)$, and $\tilde w_{ij}$ gives the normalized probability distribution.

Based on these weights, the geometry of the posterior can be visualized directly using contour plots. Unlike MCMC, this grid-based approach avoids issues of convergence diagnostics, while the two-dimensional grid enables a complete visualization of the posterior structure (e.g., unimodality, multimodality, tail behaviour). This helps to reveal parameter dependencies and assess finite-sample identifiability.

In addition to raw contour visualization, we compute highest posterior density (HPD) regions on the grid, corresponding to coverage levels analogous to one- and two-standard-deviation intervals under a Gaussian distribution (39.3\% and 86.5\%). These HPD contours provide a quantitative summary of posterior uncertainty and allow us to assess whether the joint posterior exhibits approximately normal behaviour or deviates substantially (e.g., skewness, heavy tails).

After obtaining the discretised posterior distribution, a natural next step is to extract the maximum a posteriori (MAP) estimate. Within the grid framework, the MAP corresponds to the grid point $(\lambda^*, A^*)$ with the highest weight
\begin{equation}
    (\lambda^*, A^*) = \arg\max_{\lambda,A} \; p(\lambda,A \mid \mathcal D).
\end{equation}
This provides the most plausible parameter pair, which can serve as a reference point for pseudo-data generation and subsequent model checking.



\subsubsection{Marginalization of \texorpdfstring{$A$}{A} and Consistency Check}
\label{边际化章节}
The joint posterior $p(\lambda, A \mid \mathcal D)$ captures the dependence between the two parameters. To verify theoretical consistency with the original exponential model, we consider the marginal posterior of $\lambda$ obtained by integrating out $A$.   

The central question is: does the marginal posterior of $\lambda$ coincide with that of the original exponential model (Equation~\ref{eq:16})? If so, this confirms that the extended model remains compatible with the original formulation and does not alter inference on $\lambda$.

Starting from Equation~\eqref{A_post}, we integrate out $A$
\begin{equation}
    p(\lambda \mid \mathcal D)
= \int p(\lambda, A \mid \mathcal D)\,dA.
\end{equation}
Substituting, we obtain
\begin{equation}
    p(\lambda \mid \mathcal D)\;\propto\;\pi_\lambda(\lambda)\;\int L(\mathcal D \mid \lambda, A)\,\pi_A(A)\,dA.
    \label{eq:45}
\end{equation}
Thus, the integral in~\eqref{eq:45} reduces to a constant term independent of $\lambda$. 
Importantly, this constant absorbs both the $A$-dependent part of the likelihood and the prior $\pi_A(A)$, which means that the marginal posterior of $\lambda$ is independent of the prior choice for $A$. 
Therefore,
\begin{equation}
    L(\mathcal D \mid \lambda, A)
= \lambda^{\sum_i \delta_i}\;
e^{-\lambda \sum_i y_i}\;
\Bigg[
A^{-n}\,\prod_{i:\,\delta_i=1}(A-y_i)
\Bigg]\;\mathbf 1\{A \ge \max_i y_i\}.
\end{equation}
the integral in~\eqref{eq:45} separates into a constant term independent of $\lambda$. Thus,
\begin{align}
p(\lambda \mid \mathcal{D})
&\propto \lambda^{\sum_i \delta_i}\,
e^{-\lambda \sum_i y_i}\,
\pi_\lambda(\lambda)\,
\underbrace{\int_{A \ge \max_i y_i}
A^{-n}\,\prod_{i:\,\delta_i=1}(A-y_i)\,\pi_A(A)\,dA}_{\text{constant, independent of }\lambda} \\[2pt]
&\propto \lambda^{\sum_i \delta_i}\, e^{-\lambda \sum_i y_i}\, \pi_{\lambda}(\lambda), .\\
&\propto
\text{Gamma}
\left(
\sum_{i=1}^{n} \delta_i + \alpha,\ \sum_{i=1}^{n} y_i + \beta
\right).
\end{align}
This is exactly the posterior distribution of the original exponential model (Equation~\ref{eq:16}).  

Numerically, the same result can be recovered by marginalizing over $A$ in the grid-based posterior representation: summing the normalized weights $\tilde w_{ij}$ along the $A$ dimension for each fixed $\lambda$ yields the marginal posterior of $\lambda$. This numerical implementation matches the analytical derivation above, providing a computational-theoretical cross-check.

In summary, marginalizing $A$ restores the same posterior distribution for $\lambda$ as in the original exponential model. This demonstrates that the extended model is fully consistent with the original formulation: introducing $A$ enriches the model by explicitly representing the censoring window, while preserving the fundamental inference on $\lambda$.