We analyze an employee turnover dataset~\cite{babushkin_employee_turnover} publicly shared by Edward Babushkin from a large-scale industrial survey in Russia (circa 2017). For each employee, the dataset records the duration of employment (months) and whether turnover occurred (\texttt{event}=1) or the observation was censored (\texttt{event}=0). Additional demographic and job-related variables (e.g., gender, age, industry, psychological traits) are available but used here only as background.

\textit{Key sample quantities:} \(n=1{,}129\); events \(=571\) (50.6\%); censored \(=558\) (49.4\%). Tenure: median \(=24.4\) months, IQR \(=11.7\text{–}51.3\), mean \(=36.6\), range \(=0.4\text{–}179\). For completeness: gender \(f=853\), \(m=276\); largest industries: Retail \(289\), IT \(145\), Banks \(114\).

Practically, such data inform typical tenure, time variation in turnover risk, and design effects on risk estimates. In this dissertation, it serves primarily as a motivating case, highlighting how censoring and the observation window shape inference. It is particularly suitable for this purpose because the split between events and censoring is roughly half-and-half, and the survey horizon is undocumented, making administrative censoring central to the analysis. Accordingly, we treat the observation window as part of the data-generating process and defer a formal likelihood specification until after the model-checking diagnostics.
