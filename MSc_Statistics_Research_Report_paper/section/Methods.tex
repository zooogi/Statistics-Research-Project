\paragraph{Tips for the Methods section.} In general, you may structure the main body of your report in a form that suits your project best. We recommend to start with a substantial Methods section that includes, as in a research paper, a description of the following:
\begin{itemize}
    \item Key definitions and setting using mathematical notation. Any terminology related to the topic/data needs to be clearly presented and explained to the reader.
    \item Any data used in the project should be clearly explained in sufficient detail in the Methods section so that readers can follow the rest of your report without the need to consult an other report or any of your references.
    \item If in the project you have generated simulated data, a clear presentation of the generative procedure needs to be included. 
    \item Background statistical work using mathematical notation.
    \item All maths should be presented in inline or display formulas with appropriate referencing. For example, use $\exp(x)$ not exp(x) and $\sin(\theta)$ not $sin(\theta)$ or sin($\theta$). Display formulas should be numbered using the equation environment if they are referenced in the main text. Display equation blocks should be numbered using the subequations environment:
\begin{subequations}\label{eq:Y}
    \begin{align}
        Y & = 1 + Z \label{eq:Y1} \\ 
        Z & \sim \mathcal{N}(0,1) \label{eq:Y2},
    \end{align}
\end{subequations}
    so that you can reference~\eqref{eq:Y}, ~\eqref{eq:Y1}, and ~\eqref{eq:Y2}.
    \item If your main findings are of an applied nature, you may prefer to describe any new/original statistical procedures or models here. Include the key derivations and proofs if there are any, and any supplementary derivations in the appendix. Clearly highlight your contribution relative to existing work. 
    \item If your main findings are of a theoretical nature, you may prefer to describe any new/original statistical procedures or models in the Results section.
    \item Structure your Methods section using LaTeX   subsections and reference them using labels.
    \item Aim for approximately 8-15 pages, similar in style to a general science or statistics research paper.
\end{itemize}


\subsection{My Methods subsection}\label{subsec:my_subsec}

This text is in Subsection~\ref{subsec:my_subsec}, which is a part of Section~\ref{sec:methods}.

\subsubsection{My subsubsection}
If needed, subsections can also have subsubsections within them. Keep subsections and subsubsections numbered, which will help us to reference them.


\paragraph{Paragraphs.} In the rare event that you need a more deeply nested structure within a subsubsection, the \texttt{\textbackslash paragraph\{\}} command can be useful. 